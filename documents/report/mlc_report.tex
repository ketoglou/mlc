\documentclass[12pt,a4paper,a4paper]{report}

\usepackage{graphicx}
\graphicspath{ {../images/} }
\usepackage{xltxtra}
\usepackage{xgreek}
\setmainfont[Mapping=tex-text]{GFS Didot}

\begin{document}

\title{mlc++\\ ένας απλός μεταφραστής για MIPS \\ \includegraphics[scale=0.3]{logo}}
\author{Κετόγλου Χάρης, ΑΜ:2723 \and Γκέλιας Κωνσταντίνος, ΑΜ:2669 \\{Τμήμα Μηχανικών Η/Υ και Πληροφορικής, Ιωάννινα}}
\date{Μάιος 2020}
\maketitle
\tableofcontents

\chapter{Η γλώσσα miminal++}

\section{Εισαγωγή}
Η minimal++ είναι μια απλή και μικρή γλώσσα προγραμματισμού η οποία παράγει κώδικα assembly για επεξεργαστές βασισμένους στην αρχιτεκτονική MIPS. Παρόλο που οι προγραμματιστικές της ικανότητες είναι μικρές, η εκπαιδευτική αυτή γλώσσα περιέχει πλούσια στοιχεία και η κατασκευή του μεταγλωττιστή της έχει να παρουσιάσει αρκετό ενδιαφέρον, αφού περιέχονται σε αυτήν πολλές εντολές που χρησιμοποιούνται από άλλες γλώσσες, καθώς και κάποιες πρωτότυπες. Η minimal++ υποστηρίζει συναρτήσεις και διαδικασίες, μετάδοση   παραμέτρων με αναφορά και τιμή, αναδρομικές κλήσεις και άλλες ενδιαφέρουσες δομές. Επίσης, επιτρέπει φώλιασμα στη δήλωση συναρτήσεων κάτι που λίγες γλώσσες υποστηρίζουν (το υποστηρίζει η Pascal, δεν το υποστηρίζει η C).Από την άλλη όμως πλευρά, η minimal++ δεν υποστηρίζει βασικά προγραμματιστικά εργαλεία όπως η δομή for, ή τύπους δεδομένων όπως οι πραγματικοί αριθμοί και οι συμβολοσειρές.

\section{Λεκτικές Μονάδες}
Το αλφάβητο της minimal++ αποτελείται από:\\
• τα μικρά και κεφαλαία γράμματα της λατινικής αλφαβήτου («Α»,...,«Ζ» και «a»,...,«z»),\\
• τα αριθμητικά ψηφία («0»,...,«9»    ), \\
• τα σύμβολα των αριθμητικών πράξεων («+», «-», «*», «/»), \\
• τους τελεστές συσχέτισης «<», «>», «=», «<=», «>=», «<>», \\
• το σύμβολο ανάθεσης «:=», \\
• τους διαχωριστές («;», «,», «:») \\
• καθώς και τα σύμβολα ομαδοποίησης («(»,«)»,«[»    ,«]»,«{»,«}»)\\

Τα σύμβολα "[" και "]" χρησιμοποιούνται στις λογικές παραστάσεις όπως τα σύμβολα "(" και ")" στις αριθμητικές παραστάσεις.\\

Οι δεσμευμένες λέξεις είναι:\\

\textbf{program , declare , if , else , while , doublewhile , loop , exit , forcase , incase , when , default , not , and , or , function , procedure , call , return , in , inout , input , print }\\

Οι λέξεις αυτές δεν μπορούν να χρησιμοποιηθούν ως μεταβλητές. Οι σταθερές της γλώσσας είναι ακέραιες σταθερές που αποτελούνται από προαιρετικό πρόσημο και από μία ακολουθία αριθμητικών ψηφίων.Τα αναγνωριστικά της γλώσσας είναι συμβολοσειρές που αποτελούνται από γράμματα και ψηφία,αρχίζοντας όμως από γράμμα. Ο μεταγλωττιστής λαμβάνει υπόψη του μόνο τα τριάντα πρώταγράμματα. Οι λευκοί χαρακτήρες (tab, space, return) αγνοούνται και μπορούν να χρησιμοποιηθούν με οποιονδήποτε τρόπο χωρίς να επηρεάζεται η λειτουργία του μεταγλωττιστή, αρκεί βέβαια να μην βρίσκονται μέσα σε δεσμευμένες λέξεις, αναγνωριστικά, σταθερές. Το ίδιο ισχύει και για τα σχόλια,τα οποία πρέπει να βρίσκονται μέσα στα σύμβολα /* και */ ή να βρίσκονται μετά το σύμβολο // και ως το τέλος της γραμμής. Απαγορεύεται να ανοίξουν δύο φορές σχόλια, πριν τα πρώτα κλείσουν. Δεν υποστηρίζονται εμφωλευμένα σχόλια.

\section{Μορφή προγράμματος}
\hspace{10mm}\\
program id\\
\{\\
\hspace*{10mm}declarations\\
\hspace*{10mm}subprograms\\
\hspace*{10mm}sequence of statements\\
\}\\

\section{Τύποι και δηλώσεις μεταβλητών}
Ο μοναδικός τύπος δεδομένων που υποστηρίζει η minimal++ είναι οι ακέραιοι αριθμοί. Οι     ακέραιοι αριθμοί πρέπει να έχουν τιμές από –32767 έως 32767. Η δήλωση γίνεται με την εντολή declare.Ακολουθούν τα ονόματα των αναγνωριστικών χωρίς καμία άλλη δήλωση, αφού γνωρίζουμε ότι πρόκειται για ακέραιες μεταβλητές και χωρίς να είναι αναγκαίο να βρίσκονται στην ίδια γραμμή. Οιμεταβλητές χωρίζονται μεταξύ τους με κόμματα. Το τέλος της δήλωσης αναγνωρίζεται με το ελληνικό ερωτηματικό. Επιτρέπεται να έχουμε περισσότερες των μία συνεχόμενες χρήσεις της declare.

\section{Τελεστές και εκφράσεις}
 Η προτεραιότητα των τελεστών από τη μεγαλύτερη στη μικρότερη είναι:\\
 (1) Μοναδιαίοι λογικοί: «not»\\
 (2) Πολλαπλασιαστικοί: «*», «/»\\
 (3) Μοναδιαίοι προσθετικοί: «+», «-»\\
 (4) Δυαδικοί προσθετικοί: «+», «-»\\
 (5) Σχεσιακοί «=», «<», «>», «<>», «<=», «>=»\\
 (6) Λογικό «and»\\
 (7) Λογικό «or»\\
 
 \section{Δομές τις γλώσσας}
 \subsection{Eκχώρηση}
\hspace{10mm}\\
\textbf{Id := expression }\\
 Χρησιμοποιείται για την ανάθεση της τιμής μιας μεταβλητής ή μιας σταθεράς, ή  μιας έκφρασης σε μία μεταβλητή.\\
 
 \newpage
 
 \subsection{Απόφαση if}
\hspace{10mm}\\
 \textbf{if(condition)\\\hspace*{10mm}statements1}\\ \textbf{ [else\\\hspace*{10mm}statements2] }\\
Η εντολή απόφασης if εκτιμάει εάν ισχύει η συνθήκη condition και εάν πράγματι ισχύει,τότε εκτελούνται οι εντολές statements1 που το ακολουθούν. Το else δεν αποτελεί υποχρεωτικό τμήμα της εντολής και γι’ αυτό βρίσκεται σε αγκύλη. Οι εντολές statements2 που ακολουθούν το else εκτελούνται εάν η συνθήκη condition δεν ισχύει\\
 \subsection{Eπανάληψη while}
 \hspace{10mm}\\
 \textbf{while(condition)\\\hspace*{10mm}statements}\\
 H εντολή επανάληψης while επαναλαμβάνει συνεχώς τις εντολές statements, όσο η συνθήκη condition ισχύει. Αν την πρώτη φορά που θα αποτιμηθεί η condition, το αποτέλεσμα της αποτίμησης είναι ψευδές, τότε οι statements δεν εκτελούνται ποτέ.\\    
  \subsection{Eπανάληψη loop}
 \hspace{10mm}\\
 \textbf{loop\\\hspace*{10mm}statements}\\
 H εντολή επανάληψης loop επαναλαμβάνει για πάντα τις εντολές statements. Έξοδος από το βρόχο γίνεται μόνο όταν κληθεί η εντολή \textbf{exit}.\\
 \subsection{Επανάληψη forcase}
  \hspace{10mm}\\
  \textbf{forcase\\\hspace*{10mm}(when:(condition): statements1)*\\\hspace*{10mm}default: statements2}\\
Η δομή επανάληψης forcase ελέγχει τις condition που βρίσκονται μετά τα when. Μόλις μία από αυτές βρεθεί αληθής, τότε εκτελούνται οι statements1 που ακολουθούν. Μετά ο έλεγχος μεταβαίνει στην αρχή της forcase. Αν καμία από τις when δεν ισχύει, τότε ο έλεγχος μεταβαίνει στη default και εκτελούνται οι statements2. Στη συνέχεια ο έλεγχος μεταβαίνει έξω από την forcase.
 \subsection{Επανάληψη incase}
  \hspace{10mm}\\
   \textbf{incase\\\hspace*{10mm}(when:(condition): statements)*}\\
Η δομή επανάληψης incase ελέγχει τις condition που βρίσκονται μετά τα when,     εξετάζοντας τες κατά σειρά. Για κάθε μία από αυτές που η αντίστοιχη condition ισχύει, εκτελούνται οι statements που ακολουθούν το σύμβολο “:”. Θα εξεταστούν όλες οι condition και θα εκτελεστούν όλες οι statements των οποίων οι condition ισχύουν. Αφότου εξεταστούν όλες οι when, ο έλεγχος μεταβαίνει έξω από τη δομή incase εάν καμία από τις statements δεν έχει εκτελεστεί ή μεταβαίνει στην αρχή της incase, έαν έστω και μία από τις statements έχει εκτελεστεί.\\
 \subsection{Επανάληψη doublewhile}
  \hspace{10mm}\\
\textbf{doublewhile(condition)\\\hspace*{10mm} statements1\\else\\\hspace*{10mm}statements2}\\
Tην πρώτη φορά που ο έλεγχος εισέρχεται στον βρόχο doublewhile, αποφασίζεται μέσα από την condition αν η εκτέλεση θα μεταβεί στο statements1 (true) ή αν θα μεταβεί στο statements2 (false). Από το statements1 φεύγει, όταν η συνθήκη σταματήσει να είναι true.Από το statements2 φεύγει όταν η συνθήκη σταματήσει να είναι false. Και στις δύο αυτές περιπτώσεις ο έλεγχος μεταφέρεται έξω από την δομή. Δηλαδή, δεν είναι ποτέ δυνατόν σε μία εκτέλεση της doublewhile ο έλεγχος να περάσει και από την statements1 και από την statements2.\\
Xρησιμοποιείται μέσα σε συναρτήσεις για να επιστρέφει το αποτέλεσμα της συνάρτησης.\\
 \subsection{Επιστροφή τιμής συνάρτησης}
   \hspace{10mm}\\
\textbf{return (expression)}\\
Xρησιμοποιείται μέσα σε συναρτήσεις για να επιστρέφει το αποτέλεσμα της συνάρτησης.\\
 \subsection{Έξοδος δεδομένων}
   \hspace{10mm}\\
\textbf{print (expression)}\\
Εμφανίζει στην οθόνη το αποτέλεσμα της αποτίμησης του expression.\\
 \subsection{Eίσοδος δεδομένων}
   \hspace{10mm}\\
\textbf{input (id)}\\
Ζητάει από τον χρήστη να δώσει μια τιμή μέσα από το πληκτρολόγιο.\\
 \subsection{Κλήση διαδικασίας}
   \hspace{10mm}\\
\textbf{call function\_name(actual\_parameters)}\\
Καλεί μια διαδικασία.\\
 \subsection{Έξοδος από βρόχο loop}
   \hspace{10mm}\\
\textbf{exit}\\
Εκτελεί έξοδο από βρόχο loop.\\

\newpage

\section{Υποπρογράμματα}
Η minimal++ υποστιρίζει συναρτίσεις.\\
\textbf{function id(formal\_pars)\\\{\\\hspace*{10mm}declarations\\\hspace*{10mm}subprograms\\\hspace*{10mm}statements\\\}}\\
H formal\_pars είναι η λίστα των τυπικών παραμέτρων. Οι συναρτήσεις μπορούνν να φωλιάσουν η μία μέσα στην άλλη και οι κανόνες εμβέλειας είναι όπως της PASCAL. Η επιστροφή της τιμής μιας συνάρτησης γίνεται με την return.H κλήση μιας συνάρτησης, γίνεται από τις αριθμητικές παραστάσεις σαν τελούμενο. π.χ.\\
D =  a + f(\textbf{in} x)\\
όπου f η συνάρτηση και x παράμετρος που περνάει με τιμή. Οι διαδικασίες συντάσσονται ως εξής:\\
\textbf{procedure id(formal\_pars)\\\{\\\hspace*{10mm}declarations\\\hspace*{10mm}subprograms\\\hspace*{10mm}statements\\\}}\\
H κλήση μιας διαδικασίας, γίνεται με την call.π.χ.\\
\textbf{call} f(\textbf{inout} x)\\
όπου f η διαδικασία και x η παράμετρος που περνάει με αναφορά.\\

\section{Μετάδοση παραμέτρων}
 Η minimal++ υποστηρίζει δύο τρόπους μετάδοσης παραμέτρων:\\
  • με σταθερή τιμή. Δηλώνεται με την λεκτική μονάδα in. Αλλαγές στην τιμή της δεν επιστρέφονται στο πρόγραμμα που κάλεσε τη συνάρτηση.\\
  • με αναφορά. Δηλώνεται με τη λεκτική μονάδα inout. Κάθε αλλαγή στη τιμή της μεταφέρεται αμέσως στο πρόγραμμα που κάλεσε τη συνάρτηση.Στην κλήση μίας συνάρτησης οι πραγματικοί παράμετροι συντάσσονται μετά από τις λέξεις κλειδιά in και inout, ανάλογα με το αν περνάνε με τιμή ή αναφορά.\\



\chapter{Χρήση του μεταφραστή}
Ο μεταφραστής ονομάζεται mlc και ο κώδικας του βρίσκεται στο ομώνυμο αρχείο με κατάληξη .py. Για να τρέξει ο μεταφραστής η γλώσσα python 3 χρειάζεται να είναι εγκατεστημένη στο σύστημα. Η χρήση του μεταφραστή είναι απλή :\\
\texttt{\textbf{python3 mlc.py option file.min}}\\
όπου file.min το αρχείο που περιέχει τον κώδικα του προγράμματος σε γλώσσα minimal++ και option μια παράμετρος.\\
\texttt{\textbf{option:}}\\
\texttt{\textbf{--help}} : εμφανίζει ένα κείμενο βοήθειας για τον μεταφραστή.\\
\texttt{\textbf{-save-temps}} : αποθηκεύει τα προσωρινά αρχεία που χρησιμοποιεί ο μεταφραστής.Συγκεκριμένα το αρχείο με την ενδίαμεση γλώσσα,το αρχείο με πληροφορίες για τον πίνακα συμβόλων και το αρχείο για την προσομοίωση του κώδικα σε C.\\
επίσης μπορεί να μην μπει καμία παράμετρος και έτσι ο μεταφραστής θα παράγει απευθείας το αρχείο της assembly MIPS με κατάληξη .asm.\\
Όλα τα προγράμματα της minimal++ πρέπει να είναι σε αρχεία με κατάληση .min .\\
   

\chapter{Λεκτική Ανάλυση}
\section{Πεπερασμένο Αυτόματο}
\section{Λεκτικός Αναλυτής}

\chapter{Συντακτική Ανάλυση}
\chapter{Ενδιάμεση γλώσσα}
\chapter{Πίνακας Συμβόλων}
\chapter{Errors}
\chapter{Παραγωγή κώδικα assembly MIPS}
\chapter{Δημιουργία κώδικα προσομοίωσης C}

\end{document}

